
\documentclass[11pt]{article}

\usepackage{indentfirst}
\begin{document}

\title{Shrimps and stuff}
\author{Kastners' Group}
\maketitle

\section{Preface}
	Underwater research is challenging. There are a few reasons for this: The range of frequencies used to transmit signals does not propogate well underwater. One outcome of this is that GPS, and other devices based utilizing radio frequency waves simply do not work. Another reason is that the lack of lighting makes image capture difficult. In addition, the high pressures of the ocean make it near-impossible to travel down in the ocean. 
	
	However, these significant challenges are not impossible to overcome, when it comes to researching properties of the ocean, ocean waves and the ocean floor. Researchers have developed methods and techniques to research the ocean properties. Hydrophones are underwater microphones, designed to record and listen to sound underwater. Autonomous underwater vehicles(AUVs) are underwater robots, which passively record to sound underwater, with the use of hydrophones. These robots are not attached to anything, and only arise to the surface every few days. 
	
	Scientists have realized that snapping shrimp are one of the loudest animals in the ocean, who exist on the ocean floor. These animals snap, creating a loud sound in the ocean. Scientist shave  realized that by measuring and analyzing their snaps across multiple sensors on AUVs, it is possible to determine the depth of an AUV. 

\section{Problem}

	We intend to further the research done utilizing snapping shrimp. As it stands, monitoring the position of AUV's are limited. We can only track their depth, utilizing snapping shrimp. However, using surface buoys, devices that are GPS-enabled we have much better luck tracking their position horizontal position. However, this devices are expensive, and achieving similar results with snapping shrimp seems possible. In addition, we believe that it is possible to track other properties of the ocean floor is possible through analzying the signals sent by the snapping shrimp. As sending signals from a device consumes a lot of power, it is not feasible to emit sounds from the ocean floor.  
\section{Solution}

\end{document}